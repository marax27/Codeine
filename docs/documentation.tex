\documentclass{article}
\usepackage[a4paper]{geometry}
\usepackage[utf8]{inputenc}
\usepackage{polski}
\usepackage{tabularx}
\usepackage{indentfirst}
\usepackage{multirow}
\usepackage{amssymb}
\usepackage{amsmath}
\usepackage{anysize}
\usepackage{float}
\usepackage{caption}
\usepackage{subcaption}
\usepackage{graphicx}
\usepackage{verbatim}

\usepackage{listings}
\usepackage{color}
\lstset{literate=%
{ą}{{\k{a}}}1 {ć}{{\'c}}1 {ę}{{\k{e}}}1 {ł}{{\l{}}}1 {ń}{{\'n}}1 {ó}{{\'o}}1 {ś}{{\'s}}1 {ż}{{\.z}}1 {ź}{{\'z}}1 {Ą}{{\k{A}}}1 {Ć}{{\'C}}1 {Ę}{{\k{E}}}1 {Ł}{{\L{}}}1 {Ń}{{\'N}}1 {Ó}{{\'O}}1 {Ś}{{\'S}}1 {Ż}{{\.Z}}1 {Ź}{{\'Z}}1 }

\definecolor{mygreen}{rgb}{0,0.6,0}
\definecolor{mygray}{rgb}{0.5,0.5,0.5}
\definecolor{mymauve}{rgb}{0.58,0,0.82}

\usepackage{titling}
\newcommand{\subtitle}[1]{%
	\posttitle{%
	\par\end{center}
	\begin{center}\small#1\end{center}
	\vskip0.5em}%
}

\title{Codeine - Computing over Decentralized Network, with P2P}
\subtitle{AGH University of Science and Technology\\
    Faculty of Electrical Engineering, Automatics, Computer Science and Engineering in Biomedicine}
\author{Kacper Tonia\and
        Przemysław Nocoń\and
        Kuba Komnata\and
        Sławomir Kalandyk}
\date{}

\begin{document}
%------------------------------------------------------------
\maketitle

%------------------------------------------------------------
\section{Glossary}
\begin{itemize}
    \item Agent - single application instance, is able to compute one subproblem (task) at a time
    \item Computational problem - problem solvable with Codeine. It should be divisible into a finite amount of subproblems (tasks) which can be solved independent of each other
    \item Computational network (usually referred to as "network") - a network of agents communicating with each other, who together solve one computational problem
    \item Task - a single subproblem of the computational problem
\end{itemize}

\section{Requirements}
\begin{itemize}
    \item Project should implement peer-to-peer networking on LAN
    \item We should assume that about 5 agents at once can work on our computational problem
    \item Every agent should have exactly the same application
    \item After application launch, agent should automatically connect with all other agents in the network
    \item The computational problem itself doesn't matter, it should only allow for long enough computing time to let us see the network working as intended (officially 1 hour on 5 agents, non-officially 10 - 20 mins)
    \item Task assignment should be decentralized
    \item Agents should be immune to other agents disconnecting from the network, there should be no side effects
    \item Results should be visualized, accessible (in the best case in real time)
\end{itemize}

\section{Assumptions \& Constraints}
\begin{itemize}
    \item Packet type - a seven letter long word consisting only of upper case letters (e.g. IMALIVE, NETTOPO)
    \item The solution of a single task should be able to fit in a single UDP packet (<64kB)
    \item Every task has it's ID and immutable State, common for all tasks
    \item Every task can be solved
\end{itemize}

\section{Networking}
\subsection{Packets}

\begin{itemize}
    \item Topology discovery, registering agents, agent check
    \begin{itemize}
        \item IMALIVE - I am alive \verb!<>!
        \item NETTOPO - send network topology \verb!<agent[]>!
    \end{itemize}
    \item Task assignment
    \begin{itemize}
        \item REGTASK - register new task \verb!<task_id>!
        \item STOPWIP - don't start: this task is work-in-progress \verb!<task_id>!
        \item STOPLOW - don't start: lower priority \verb!<task_id>!
    \end{itemize}
    \item Result distribution
    \begin{itemize}
        \item TASKRES - send task result \verb!<task_id, task_result>!
    \end{itemize}
    \item Confirmation
    \begin{itemize}
        \item ACKNOWL - acknowledge \verb!<previous_package_control_sum>!
    \end{itemize}
\end{itemize}

\subsection{Rules}
\begin{itemize}
    \item IMALIVE $\rightarrow$ NETTOPO (optional)
    \item REGTASK $\rightarrow$ STOPWIP $\vert$ STOPLOW $\vert$ TASKRES
    \item TASKRES $\rightarrow$ ACKNOWL
\end{itemize}

\subsection{Network scenarios}
\subsubsection{Scenario 1}
\noindent\textbf{Story:} \\
Agent tries to join the network right after launching Codeine. \\\\
\textbf{Prerequisites:}
\begin{itemize}
    \item None
\end{itemize}
\textbf{Scenario:}
\begin{enumerate}
    \item Agent broadcasts IMALIVE packet.
    \item Agent waits for ??? seconds for replies from every other agent already in the network.
    \item Every other agent already in the network replies with NETTOPO packet.
    \item Agent registers the network. End of Scenario 1.
\end{enumerate}
\textbf{Scenario extensions:}
\begin{itemize}
    \item[3a.] There is no response from the network (proceed to calculations). End of Scenario 1.
\end{itemize}

\subsubsection{Scenario 2}
\noindent\textbf{Story:} \\
Agent periodically informs the network that he's still alive. \\\\
\textbf{Prerequisites:}
\begin{itemize}
    \item Agent is already in the network
\end{itemize}
\textbf{Scenario:}
\begin{enumerate}
    \item Agent broadcasts IMALIVE packet. End of Scenario 2.
\end{enumerate}

\subsubsection{Scenario 3}
\noindent\textbf{Story:} \\
Agent wants to register a task \\\\
\textbf{Prerequisites:}
\begin{itemize}
    \item Agent is already in the network
\end{itemize}
\textbf{Scenario:}
\begin{enumerate}
    \item Agent broadcasts REGTASK packet.
    \item Agent waits for ??? seconds for replies.
    \item Every other agent registered in his net topology base replies with CARRYON.
    \item Agent starts calculations on that task. End of Scenario 3.
\end{enumerate}
\textbf{Scenario extensions:}
\begin{itemize}
    \item[3a.] There is no response from the network.  
    \begin{itemize} 
    \item[3a.1.] Agent starts calculations on that task. End of Scenario 3.
    \end{itemize}
    \item[3b.] An agent replies with STOPWIP.
    \begin{itemize} 
    \item[3b.1.] Agent sets that task's state to WIP.
    \item[3b.2.] Agent chooses another task. Repeat from point 1. End of Scenario 3.  
    \end{itemize}
    \item[3c.] An agent replies with STOPLOW.  
    \begin{itemize} 
    \item[3c.1.] Agent chooses another task. Repeat from point 1. End of Scenario 3.  
    \end{itemize}
    \item[3d.] An agent replies with TASKRES. 
    \begin{itemize} 
    \item[3d.1.] Agent registers received task result. 
    \item[3d.2.] Agent chooses another task. Repeat from point 1. End of Scenario 3.  
    \end{itemize}
\end{itemize}

\subsubsection{Scenario 4}
\noindent\textbf{Story:} \\
Agent wants to broadcast task results. \\\\
\textbf{Prerequisites:}
\begin{itemize}
    \item Agent is already in the network
    \item Agent has calculated a task and received a concrete result
\end{itemize}
\textbf{Scenario:}
\begin{enumerate}
    \item Agent broadcast TASKRES packet.
    \item Agent waits for ??? seconds for replies.
    \item Every other agent registered in his net topology base replies with ACKNOWL. End of Scenario 4.
\end{enumerate}
\textbf{Scenario extensions:}
\begin{itemize}
    \item[3a.] If any agent hasn't replied with ACKNOWL, send packet TASKRES again to him. Repeat ??? times. 
    \begin{itemize}
        \item[3a.1.] An agent hasn't replied after ??? sent TASKRES packets. Remove him from net topology base.
    \end{itemize} 
\end{itemize}

\subsubsection{Scenario 5}
\noindent\textbf{Story:} \\
Agent has received a TASKRES packet with task ID of a task he already has a result of. \\\\
\textbf{Prerequisites:}
\begin{itemize}
    \item Agent is already in the network
    \item Agent already has results of at least one task
\end{itemize}
\textbf{Scenario:}
\begin{enumerate}
    \item Agent received a TASKRES packet.
    \item Agent tries to register the task result. Task with that ID already has a registered result.
    \item Task result of received packet is ignored. 
    \item Reply with ACKNOWL packet. End of Scenario 5.
\end{enumerate}

\subsubsection{Scenario 6}
\noindent\textbf{Story:} \\
Agent A tries to register a task with ID == X. Agent B replies with STOPWIP packet. Agent A sets task X's state to WIP. Agent B then disconnects. \\\\
\textbf{Prerequisites:}
\begin{itemize}
    \item Agent is already in the network
\end{itemize}
\textbf{Scenario:}
\begin{enumerate}
    \item Agent A broadcasts REGTASK packet with task ID == X.
    \item Agent B replies with STOPWIP.
    \item Agent A sets task X's state to WIP.
    \item Agent B disconnects.
    \item Agent A doesn't receive IMALIVE packets from agent B for ??? minutes.
    \item Agent A sets task X's state to unregistered/free/not-yet-completed. End of Scenario 6.
\end{enumerate}
\textbf{Scenario extensions:}

\end{document}